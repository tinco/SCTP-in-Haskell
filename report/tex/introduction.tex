\chapter*{Introduction}
\addcontentsline{toc}{chapter}{Introduction}
This document is a report on an exploration of two functional programming techniques for dealing with side effects.
The first technique is called monadic I/O, the second is called stream-based I/O.
The most popular Haskell implementation, Glasgow Haskell Compiler (GHC) \cite{haskell_implementations}, has favored the monadic technique over the stream-based technique \cite{hudak_history_2007}.
This report shows if it can still be relevant to work with side effects in a stream-based way. 

The topic of the exploration is the implementation of a transport layer network protocol by the name of Stream Control Transmission Protocol (SCTP) as specified by the Internet Engineering Task Force (IETF).

The implementation of this protocol shows how each of the researched techniques can be used to handle the various side effects SCTP requires to function. 

In the first chapter the goals of the report are detailed. After that the technologies and the reason for their selection are explained. In the third chapter the report goes into the ideas and history of I/O in Haskell.
The report elaborates on the exploration process in the fourth chapter. Related work is explored in the fifth chapter.
In the last chapter the conclusion of the research is presented.
