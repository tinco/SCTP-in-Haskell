\chapter{I/O}
Eventually in every network protocol implementation data has to been sent to and received from the underlying stack.
The content of the received data is not necessarily what is expected and the underlying stack most certainly has side effects.
How the language deals with these impurities influences its expressiveness and complexity.
It can even influence the architecture of the network protocol implementation.
Communication with the underlying stack usually is done using the I/O libraries and syntax of the language.
This is why in this chapter we will first explore their basic functionality.
\section{History of I/O in Haskell}
Haskell has known three approaches to I/O, stream-based, continuation-based and monadic.
\subsection{Stream-based I/O}
In stream based I/O an application is modeled as a stream of inputs mapping to a stream of outputs. Examples are taken from \cite{hudak_history_2007} and \cite{hudak_conception_1989}, sometimes slightly altered where deemed necessary.

\begin{lstlisting}[caption={Mapping of responses to requests}]
type Behaviour = [Response] -> [Request]
\end{lstlisting}

In this style the application generates requests for the operating system, to which the operating system generates appropriate responses. The requests are constructed as a regular Haskell list. An advantage of this approach is that the language does not need any special syntactic addition reading and appending to lists already is a core feature of Haskell.

\begin{lstlisting}[caption={Example of stream-based I/O in Haskell}]
main :: Behaviour
main ~(Success : ~((Str userInput) : ~(Success : ~(r4 : _))))
	= [
		AppendChan stdout "enter filename\n",
		ReadChan stdin,
		AppendChan stdout name,
		ReadFile name,
		AppendChan stdout
			(case r4 of
				Str contents -> contents
				Failure err -> "Can't open file")
		]
	where (name : _) = lines userInput
\end{lstlisting}

In this listing the '\textasciitilde' operator is used to indicate lazy pattern matching, this means that Haskell will attempt to match only when the values of the pattern are to be evaluated. This allows the programmer to match on the return value of the requests he issues in the function itself, using them for the further actions within that function.

\subsection{Continuation-based I/O}
In Haskell's continuation based style the application is still modeled as a stream of inputs mapping to a stream of outputs, but the lazy pattern matching is replaced by continuations. Continuations are functions that are passed as arguments to request functions and are called with the response value as parameter. The following source code is an example of a continuation-based readFile method implemented using stream-based I/O.

\begin{lstlisting}[caption={Definition of readFile in continuation-based I/O in Haskell}]
readFile :: Name-> FailCont -> StrCont -> Behaviour
readFile name fail succ ~(response:responses)
	 =
		ReadFile name : case response of
			Str val -> succ val responses
			Failure msg -> fail msg responses
\end{lstlisting}

This style was preferred over stream-based I/O by many Haskell programmers because the control logic is closer to the inputs.

\begin{lstlisting}[caption={Example of continuation-based I/O in Haskell}]
main :: Behaviour
	main =
 		appendChan stdout "enter filename\n" abort (
		readChan stdin abort (\userInput ->
		letE (lines userInput) (\(name : _) ->
		appendChan stdout name abort (
		readFile name fail (\contents ->
		appendChan stdout contents abort done)))))
	where
		fail ioerr =
			appendChan stdout "can't open file" abort done
abort	:: FailCont
abort err resps = []
letE	:: a->(a->b)->b
letE x k = k x
\end{lstlisting}

\subsection{Monadic I/O}
Sometime after Haskell's first release the relevance of monads to programming languages was discovered and in particular to functional programming languages. It was discovered that the monad could be used to express sequentially ordered computations. This proved to be ideal for expressing I/O.

Monads are a class of types for which two functions are defined, the first is the return function, which constructs a monad from a value.
 The second is the bind function, it takes a monad, and a function that takes a value and returns a monad and returns a monad.
The bind function has the \lstinline{>>=} operator in Haskell.
 This is a bit abstract, but among other things the monad can be used to model a sequence of computations like this:

\begin{lstlisting}[caption={Sequence of computations monad}]
data Monad a = M a
return :: a -> Monad a
return x = M x
(>>=) :: (Monad a) -> (a -> Monad b) -> Monad b
(>>=) (M x) f = f x
\end{lstlisting}

In this implementation the bind function does nothing but unwrap its parameter value, taking it from the monad, and call the function with it.
A monad is constructed by chaining monadic functions, functions that return a monad, together using the bind function. Since each monadic function takes an unwrapped value as a parameter and returns a wrapped value it is guaranteed that when unwrapping a monad every function in the chain is evaluated in sequential order.

The I/O monad makes handy use of this property. This guaranteed ordering makes it safe to reason about functions that have side effects, as long as they are in the IO monad. 
A token which represents the state of the world along from one monadic function to the next.

\begin{lstlisting}[caption={Example of monadic I/O in Haskell}]
main :: IO ()
main =
	appendChan stdout "enter filename\n" >>
	readChan stdin >>= \userInput ->
	let (name : _) = lines userInput in
	appendChan stdout name >>
	catch (readFile name >>= \contents ->
		appendChan stdout contents)
		(appendChan stdout "can't open file")
\end{lstlisting}

In this example the \lstinline!>>! chains together two actions, the \lstinline!>>=! operator does the same, but also makes the return value of the previous action available to the subsequent actions.

This approach became so popular it became the de facto standard even before it replaced the stream- and continuation-based approach in the 1.3 release of Haskell\cite{haskell12to13}. A syntactic sugar was also introduced reducing the previous example to:

\begin{lstlisting}[caption={Example of monadic I/O in Haskell}]
main :: IO ()
main = do
	appendChan stdout "enter filename\n"
	userInput <- readChan stdin
	let (name : _) = lines userInput
	appendChan stdout name
	catch (do contents <- readFile name
		appendChan stdout contents)
		(appendChan stdout "can't open file")
\end{lstlisting}
