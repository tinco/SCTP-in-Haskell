\chapter{Conclusion}
Having implemented the SCTP protocol in both a monadic and a stream-based I/O style and having analyzed their advantages and disadvantages we can make recommendations based on their merits.

The stream-based approach was deserted by the Haskell Committee and community because the distance between actions and their responses. The problems this causes showed clearly in our implementation where a lot of complexity went into recombining the events with the handlers they were meant for. 

With the complexity comes an advantage, the distance provides us with effortless concurrency. Actions are executed in parallel by default in this implementation and the event loop reliefs the programmer of worrying about the details of safe concurrent I/O. 

But when implementating the SCTP network protocol this advantage is not exploited as the I/O actions required are not particularily expensive.
Forcing every action to be performed in a different thread also has penalties. The locks used to achieve the sequencing have a significant overhead which could shadow the time gained in executing the I/O concurrently.

The event channel style could be used in the monadic style too, by using a library that defines the functions, but since it is not part of the runtime system some responsibility would still lie with the programmer, reducing the effort but not making it effortless.

The monadic I/O style is dominant in Haskell, which is attributed to its ease of use. The binding operators of the monad allow the programmer to work on the results of actions right where the actions are called whilst still safely executing the actions in dictated order.

The monad does not fully protect the programmer from concurrency problems, but does require the programmer to be explicit about concurrency.
	The concurrency primitives it offers are very easy to use but do introduce mutable state into the program, and puts the programmer at risk of making a mistake and leaving room for deadlocks or starvation.

There is work being done on implementing software transactional memory and introducing new concurrency primitives for the monadic I/O system that reduce the risks of deadlocks significantly.

Although stream-based I/O has some promising properties it looks like the Haskell Committee was right in deciding to replace it with the monadic style. Even when refreshing the style with some new features, the disadvantages of the stream-based style outweigh its advantages, at least in the implementation of the SCTP network protocol.

To expand the knowledge of this domain further research could be done into the continuation passing style, which is another interesting I/O technique and in syntactic sugar for the stream-based style which might benefit greatly from reduced programming complexity and might work very well with the new concurrency primitives proposed.
